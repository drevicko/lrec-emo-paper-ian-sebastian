\documentclass[10pt, a4paper]{article}
\usepackage{lrec2016}
\usepackage{coloremoji}
% https://github.com/alecjacobson/coloremoji.sty
% \usepackage{multibib}
% \newcites{languageresource}{Language Resources}
\usepackage{graphicx}
% for eps graphics
% \usepackage[utf8]{inputenc}

\usepackage{epstopdf}
% \usepackage[latin1]{inputenc}
% Confirmation Number:	16
% Submission Passcode:	16X-E3E3G2G7P2
 
\title{Emoji as Emotion Tags for Tweets}

\name{Ian D. Wood, Sebastian Ruder}

\address{Insight Centre for Data Analytics; Aylien Ltd.\\
         National University of Ireland, Galway; Dublin, Ireland\\
% * <subscriptions@drevicko.com> 2016-03-25T15:37:45.429Z:
% ^.
         \texttt{firstname.lastname@insight-centre.org}}

\abstract{
In many natural language processing tasks, supervised machine learning approaches have proved most effective, and substantial effort has been made into collecting and annotating corpora for building such models.
Emotion detection from text is no exception; however, research in this area is in its relative infancy, and few emotion annotated corpora exist to date. A further issue regarding the development of emotion annotated corpora is the difficulty of the annotation task 
and resulting inconsistencies in human annotations.
One approach to address these problems is to use self-annotated data, using explicit indications of 
emotions included by the author of the data in question.
This approach has had success with using text emoticons and selected hash tags for sentiment annotation~\cite{Davidov2010Enhanced}
and emotion-specific hash tags for emotion annotation~\cite{Mohammad2012Emotional,Mohammad2015Using}.
We present a study of the use of unicode emoji as self-annotation of a Twitter user's emotional state.
Emoji are found to be used far more extensively than hash tags and we argue that they present a more faithful representation 
of a user's emotional state.
We present an evaluation plan including
% We evaluate our data by 
training a supervised emotion detection model and applying it to the SemEval2007 data set as well as 
manual annotation of a subset of collected tweets.
% We propose a similar approach using unicode emoji and argue that this approach is in many ways superior to using hash tags.
\newline 
\Keywords{Twitter, hash tags, emotion annotation, emotion detection, emoji, emoticons} }

\begin{document}

\maketitleabstract

\section{Previous Work}
A previous study examined emoji in combination with selected hash tags and emoticons to build statistical models of emotion based on Plutchik's emotion scheme, which has 8 emotions in pairs of emotional opposites~\cite{Suttles2013Distant}. 

\section{Emotion Expression in Text Only Communication}
\label{sec:emotion_expression_in_text_only_communication}
Facial expressions, voice inflection and body stance are all significant communicators of emotion~\cite{Johnston2015Apa}. 
Indeed research into emotion detection from video and voice has found that arousal (the level of excitement or activation associated with an emotional experience) is 
difficult to detect in text transcripts, implying that those aspects are not strongly expressed in text.
One might think, therefore, that text-only communication would be emotion-poor, containing less expression of emotion than face to face or vocal communication.

Research into text-only communication has, however, found that people find ways to communicate emotion, despite the lack of face, voice and body stance, and that text only communication is no less rich in emotional content than face to face communication~\cite{Derks2008Role}.
% This is promising news for research into emotion in online social media.
Other research has found that text emoticons (text sequences that indicate facial expressions, such as ``(-:'') produce similar brain responses to faces~\cite{Churches2014Emoticons}, and it is not unreasonable to expect that facial expression emoji (unicode characters that whose glyphs are small images, such as ``😅'') to function similarly.

In recent years, marketing researchers have observed significant and continuing increases in the use of emoji in online media. This increase was not constrained to young internet users, but across all ages. 
Facial expression emoji have become a common method for emotion communication in online social media that appears to have wide usage across many social contexts, and are thus excellent candidates for the detection of emotions and author-specified labelling of text data.

%TODO: expand emoji proliferation and psychological usage section 

\section{Collecting Emoji Tweets}
A selection of commonly used emoji\footnote{as indicated by http://emojitracker.com/} with clear emotional content were hand selected as emotion indicators and tweets that contained at least one of the selected emoji were collected. We used Ekman's emotion classification of 6sxbabsic emotions~ \cite{Ekman1992Argument}f  r our experiments .T
Another common scheme for categorical emotion classification was presented by Plutcik~\cite{plutchik1980general} and includes two extra basic emotions, trust and anticipation.
There were no emoji we considered clearly indicative of these emotions and Pevious research found \cite{Suttles2013Distant} that emoji are few and unreliable for the o
hT eselected emoji are indicated in Tables~\ref{tab:selected-emoji} and~\ref{tab:selected-emoji-codpoints}.

\begin{table}
	\begin{tabular}{ll}
		joy & 😀😂😃😄😆😇😉😊😋😌😍😎😏🌞☺😘😜😝😛 \\
		& 😺😸😹😻😼❤💖💕😁♥ \\
		anger & 😬😠😐😑😠😡😖😤😾 \\
		disgust & 💩 \\
		fear & 😅😦😧😱😨😰🙀 \\
		sad & 😔😕☹😫😩😢😥😪😓😭😿💔 \\
		surprise & 😳😯😵😲 \\
	\end{tabular}
	\caption{Selected Emoji}
	\label{tab:selected-emoji}
\end{table}

\begin{table}
	\begin{tabular}{ll}
joy      & {\tiny U+1F600, U+1F602, U+1F603, U+1F604, U+1F606, U+1F607, U+1F609, U+1F60A, U+1F60B} \\
         & {\tiny U+1F60C, U+1F60D, U+1F60E, U+1F60F, U+1F31E, U+263A, U+1F618, U+1F61C, U+1F61D} \\
         & {\tiny U+1F61B, U+1F63A, U+1F638, U+1F639, U+1F63B, U+1F63C, U+2764, U+1F496, U+1F495} \\
         & {\tiny U+1F601, U+2665} \\
anger    & {\tiny U+1F62C, U+1F620, U+1F610, U+1F611, U+1F620, U+1F621, U+1F616, U+1F624, U+1F63E} \\
disgust  & {\tiny U+1F4A9} \\
fear     & {\tiny U+1F605, U+1F626, U+1F627, U+1F631, U+1F628, U+1F630, U+1F640} \\
sad      & {\tiny U+1F614, U+1F615, U+2639, U+1F62B, U+1F629, U+1F622, U+1F625, U+1F62A, U+1F613} \\
         & {\tiny U+1F62D, U+1F63F, U+1F494} \\
surprise & {\tiny U+1F633, U+1F62F, U+1F635, U+1F632} \\
	\end{tabular}
	\caption{Selected Emoji Code Points}
	\label{tab:selected-emoji-codpoints}
\end{table}

There are a few choices and difficulties in selecting these emoji that should be noted.
First, it was difficult to identify emoji that clearly indicated disgust. 
An emoji image with green vomit has been used in some places, including Face Book, however this is not part of the Unicode official emoji set (though is slated for release in 2016) and does not currently appear in Twitter.

\begin{table*}[!ht]
\centering
	\begin{tabular}{c | r | r | r | r | r | r | r}
\textbf{Language} & \textbf{Total} & \textbf{Joy} & \textbf{Sadness} & \textbf{Anger} & \textbf{Fear} & \textbf{Surprise} & \textbf{Disgust}  \\\hline
en  & 190,591 & 136,623 & 36,797 & 7,658 & 6,060 & 2,943 & 510 \\
ja  & 99,032 & 68,215 & 17,397 & 4,595 & 4,585 & 3,631 & 609 \\
es  & 65,281 & 45,809 & 11,773 & 3,877 & 2,532 & 1,176 & 114 \\
UNK  & 56,597 & 42,535 & 9,217 & 1,959 & 1,624 & 1,033 & 229\\
ar  & 44,026 & 29,976 & 11,216 & 1,114 & 1,084 & 5,72 & 64 \\
pt  & 29,259 & 21,987 & 4,894 & 1,208 & 8,89 & 233 & 48 \\
tl  & 20,438 & 14,721 & 4,096 & 752 & 656 & 176 & 37 \\
in  & 18,910 & 13,578 & 3,175 & 1,018 & 738 & 323 & 78 \\
fr  & 13,848 & 10,567 & 1,821 & 651 & 572 & 213 & 24\\
tr  & 8,644 & 6,935 & 773 & 419 & 305 & 201 & 11\\
ko  & 7,242 & 5,980 & 916 & 142 & 113 & 87 & 4\\
ru  & 5,484 & 4,024 & 646 & 411 & 317 & 74 & 12\\
it  & 4,086 & 3,391 & 376 & 156 & 119 & 34 & 10 \\
th  & 3,828 & 2,461 & 857 & 227 & 156 & 124 & 3\\
de  & 2,773 & 2,262 & 235 & 119 & 81 & 69 & 7
\end{tabular}
	\caption{Number of collected tweets per emoji for the top 15 languages (displayed with their ISO 639-1 codes). UNK: unknown language.}
	\label{tbl:top-ten-langs}
\end{table*}

The second difficulty concerns the interpretation and popular usage of emoji. 
All emoji have an intended interpretation (indicated by their description in the official unicode list), however it is not guaranteed that their popular usage will align with the description. 
The choices made in this study were intended as a proof of concept, drawing on the personal experiences of a small group of people. 
Though these choices are likely to be, on the whole, reasonably accurate, 
A more thorough analysis of 
% It would be of some interest to study 
emoji usage through the analysis of associated words and contexts is in order. 
% Such a study could establish a more empirical estimate of the actual emotional content of emoticons as publicly used. 

The ``sample'' endpoint of the Twitter public streaming API was used to collect tweets. This endpoint provides a random sample of all tweets produced in twitter.
Tweets containing at least one of the selected emoji were retained. The ``sample'' endpoint is not an entirely unbiased sample, with a substantially smaller proportion of all tweets sampled during times of high traffic~\cite{Morstatter2013Is}. This was considered to be of some benefit for this study, as it reduces the prominence of individual significant effects and associated bias in the collected data.

Twitter has a streaming endpoint to which search phrase can be supplied (the``filter'' endpoint), providing only tweets that mach the search criteria.
This endpoint has two disadvantages: first, you can only search on whole (whitespace delimited) words, and not individual unicode glyphs, thus only emoticons surrounded by whitespace are retrieved. 
Second the volume of data for an emoji search phrase build from the above list is very large. A substantial data set could be collected in a single day, however this would be subject to biases resulting from the particular trending topics during that day. 
Collecting a smaller proportion of tweets over a longer period mitigates this problem.
% The ``filter'' endpoint was not used
As an illustration of the trending topic problem, in our initial experiment with the ``filter'' endpoint, the most common hash tag (in 35 thousand tweets) was ``\emph{\#mrandmrssotto}'', which relates to a prominent wedding in the US Philipino community.

We also considered a set of emotion-related hash tags (similar to in~\cite{Mohammad2012Emotional}), however we found that the number of such tweets was orders of magnitude less than tweets with our emotion emoji. That combined with the evidence from psychology that connects emoji to emotion expression (see Section~\ref{sec:emotion_expression_in_text_only_communication}) prompted us to focus on emoji for this study.

\section{Data Summary}
Over a two week collection period, we collected a total of over half a million tweets, of which 190,591 were tagged by Twitter as English.
Note that all the tweet counts provided below do not include retweets. 
These are considered to bias the natural distribution of word frequencies due to the power-law distribution of retweet frequencies and the fact that a retweet contains verbatim text from the original tweet.

\section{Evaluation}

We carry out two forms of evaluation: Firstly, in section \ref{sec:classifier_evaluation}, we evaluate the quality of classifiers trained using emoji-labeled data; secondly, in section \ref{sec:emoji_evaluation}, we evaluate the quality of the chosen emojis as emotion indicators.

\subsection{Evaluation of classifiers} \label{sec:classifier_evaluation}
First, we will train support vector machine models on the collected data for each basic emotion. 
Cross-validation will be carried out on the collected data and a final model will be applied to the SemEval2007 data set~\cite{Strapparava2007Semeval2007} then compared to results from that competition as well as cross validation results for the same model trained on SemEval2007 data.
Features for these model will include word, ngram, hash tag and emoji tokens as well as emotion scores calculated with the hash tag emotion lexicon~\cite{Mohammad2012Emotional}. Note that there is evidence that term usage in social media can be useful for detection of emotions in media such as news articles~\cite{Mohammad2012Emotional}.

{\em Note to reviewers}: It is unfortunate that we have not been able to perform these evaluations prior to writing this abstract, however we submit in any case due to the important prospect of new and easily obtainable labelled data. It is also likely that during the coming week, at the end of which final papers for the workshop should be completed, that we will complete the automated evaluation and a small manual evaluation. If not, I propose to withdraw the paper and wait for a future venue. This would be a shame, as this workshop would seem the most appropriate venue for this work.

\subsection{Evaluation of emojis} \label{sec:emoji_evaluation}

For the second evaluation, we selected a random subset of 360 of the collected English tweets. For these, we removed emotion-indicate emojis and created an annotation task asking the annotator to annotate all emotions expressed in the text.

In past research using crowd-sourcing, usually three annotators annotate a tweet. As emotion annotation is notoriously ambiguous, we increased the number of annotators. In total, 17 annotators annotated between 60 and 360 of the provided tweets, providing us with a large sample of different annotations.

For calculating inter-annotator agreement, we use Fleiss' kappa as this (in contrast to Cohen's kappa) allows us to take into account (parital) annotations by more than two annotators. We weight each annotation with $ 6 / n_{ij}$ where $n_{ij}$ is the number of emotions annotated by annotator $i$ for tweet $j$ in order to prevent a bias towards annotators that favor multiple emotions. This yields $\kappa$ of 0.51, which signifies moderate agreement, a value in line with previous reported research.

To gain an understanding of the correlation, between emoti% * <subscriptions@drevicko.com> 2016-03-29T09:55:15.097Z:
%
% could remove explanation of Fleiss kappa if we need to save space
%
% ^.
ons and emojis, we calculate PMI scores between emojis and emotions. We first calculate PMI scores between emojis and the emotion chosen by most annotators per tweet (scores arethe ilar fo witeothe selecvtrys, which we show in Table \ref{tab:pmi_results_top_emotions}.

\begin{table}[!ht]
\centering
\begin{tabular}{c | r | r | r | r% * <subscriptions@drevicko.com> 2016-03-29T09:59:45.880Z:
%
% we could look for an easier to follow description of the "emotion by vote" vs "all emotion annotations" schemes..
%
% ^.
 | r | r | r}
 & \textbf{Joy} & \textbf{Dis.} & \textbf{Sur.} & \textbf{Fear} & \textbf{Sad.} & \textbf{Ang.} & \textbf{\O} \\\hline
\textbf{Joy} & \textbf{.40} & -.53 & .08 & -.59 & -.59 & -.62 & -.12 \\
\textbf{Dis.} & .01 & \textbf{.33} & -.11 & -.02 & -.24 & -.27 & .17 \\
\textbf{Sur.} & -.49 & .31 & \textbf{.64} & -1.00 & -.03 & -.29 & .15 \\
\textbf{Fear} & .12 & -.16 & -.12 & \textbf{.66} & -.14 & -.07 & -.03 \\
\textbf{Sad.} & .11 & -.68 & -.58 & \textbf{.76} & .66 & -.37 & -.69 \\
\textbf{Ang.} & -.58 & .71 & -.22 & -.13 & -.35 & \textbf{.87} & .06
\end{tabular}
\caption{PMI scores between emojis and emotions chosen by most annotators per tweet. Emoji $\downarrow$, emotion $\rightarrow$. \O: No emotion.}
\label{tab:pmi_results_top_emotions}
\end{table}

Note that among all emojis, emotions are correlated most highly with their corresponding emojis. Anger and -- to a lesser degree -- surprise emojis are also correlated with disgust, while we observe a high correlation between sadness emojis and fear. Additionally, some emojis that we have associated with sadness and fear seem to be somewhat ambiguous, showcasing a slight correlation with joy.

Calculating PMI scores not only between emojis and those emotions, which have been selected by the most annotators for each tweet, but all selected emotions produces a slightly different picture, which we show in Table \ref{tab:pmi_results_all_emotions}.

\begin{table}[!ht]
\centering
\begin{tabular}{c | r | r | r |% * <subscriptions@drevicko.com> 2016-03-29T10:01:02.986Z:
%
% as above ... the emotion selection schemes could be described more clearly
%
% ^.
 r | r | r | r}
 & \textbf{Joy} & \textbf{Dis.} & \textbf{Sur.} & \textbf{Fear} & \textbf{Sad.} & \textbf{Ang.} & \textbf{\O} \\\hline
\textbf{Joy} & \textbf{.32} & -.35 & .04 & -.24 & -.56 & -.46 & -.27 \\
\textbf{Dis.} & -.17 & \textbf{.27} & -.36 & -.14 & .09 & .11 & .17 \\
\textbf{Sur.} & -.23 & .20 & .35 & \textbf{.63} & -.27 & -.13 & -.03 \\
\textbf{Fear} & .23 & -.31 & .29 & \textbf{.31} & .16 & -.20 & .22 \\
\textbf{Sad.} & .16 & -.33 & -.08 & -.13 & \textbf{.26} & -.16 & -.57 \\
\textbf{Ang.} & -.50 & .48 & -.15 & .09 & .21 & \textbf{.61} & .06
\end{tabular}
\caption{PMI scores between emojis and all annotated emotions. Emoji $\downarrow$, emotion $\rightarrow$. \O: No emotion.}
\label{tab:pmi_results_all_emotions}
\end{table}

The overall correlations still persist; an investigation of scores where the sign has changed reveals new insights: Surprise and fear are closely correlated now, with surprise emojis showing a strong correlation with fear, while fear emojis are correlated with surprise. This interaction wasn't evident before, having been eclipsed by the prevalence of fear and sadness. Additionally, disgust emojis now show a slight correlation with sadness and anger, fear emojis with sadness, and anger emojis with fear and sadness.

Finally, we calculate precision, recall, and F1 using the emojis contained in each tweet as predicted labels. We calculate scores both using the emotion chosen by most annotators per tweet (as in Table \ref{tab:pmi_results_top_emotions}) and all emotions (as in Table \ref{tab:pmi_results_all_emotions}) as gold label and show results in Table \ref{tab:precision_emojis}.

\begin{table}[!ht]
\centering
\begin{tabular}{l | c | c | c | c | c | c}
\textbf{Emotion} & \textbf{P$_{top}$} & \textbf{R$_{top}$} & \textbf{F1$_{top}$} & \textbf{P$_{all}$} & \textbf{R$_{all}$} & \textbf{F1$_{all}$} \\\hline
Joy & 0.51 & 0.45 & 0.48 & 0.67 & 0.41 & 0.51 \\
Disgust & 0.13 & 0.24 & 0.17 & 0.33 & 0.21 & 0.26 \\
Surprise & 0.24 & 0.33 & 0.28 & 0.57 & 0.29 & 0.38 \\
Fear & 0.03 & 0.33 & 0.06 & 0.13 & 0.24 & 0.17 \\
Sadness & 0.32 & 0.45 & 0.38 & 0.33 & 0.17 & 0.22 \\
Anger & 0.21 & 0.45 & 0.28 & 0.39 & 0.19 & 0.25
\end{tabular}
\caption{Precision, recall, and F1 scores for emojis predicting annotated emotions. $_{top}$: emotion selected by most annotators used as gold label. $_{all}$: all emotions chosen by annotators used as gold labels.}
\label{tab:precision_emojis}
\end{table}

As we can see, joy emojis are the best at predicting their corresponding emotion, while fear is generally the most ambiguous. Fear emojis are present in many more tweets that are predominantly associated in fear and even when taking into account weak associations, only about every eighth tweet containing a fear emoji is also associated with fear. Disgust, anger, and sadness are similarly present in only about every third tweet containing a corresponding emoji, although sadness usually dominates when it is present. While surprise is less often the dominating emotion, its emojis are the second-best emotion indicators in tweets.

\section{Conclusion}
% We present an approach that promisesPbegin{table}[]

We have collected a substantial and multilingual data set of tweets containing emotion-specific emoji in a short time. We argue that we can expect these emoji to perform well as ground truth indicators of tweet emotion content and propose evaluations of that claim. 
The lack of large, quality annotated data for emotion detection in social media and other text is a substantial barrier to continued research efforts in that area, and the approach presented here promises to provide some relief.

\section*{Acknowledgements}

This publication has emanated from research supported by a research grant from Science Foundation Ireland (SFI) under Grant Number SFI/12/RC/2289.
This project has emanated in part from research conducted with the financial support of the Irish Research Council (IRC) under Grant Number EBPPG/2014/30 and with Aylien Ltd. as Enterprise Partner. 

% \nocite{*}
\section{Bibliographical References}
\label{m% * <subscriptions@drevicko.com> 2016-03-29T10:05:01.606Z:
%
% we should acknowledge MixedEmotions also
%
% ^.
ain:ref}

\bibliographystyle{lrec2016}
% \bibliography{xample}
\bibliography{Remote}
% \bibliographylanguageresource{xample}

\end{document}
